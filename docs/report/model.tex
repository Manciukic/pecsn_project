\section{Model}

\subsection{Description}
\begin{figure}[h!]
    \centering
    \includegraphics[width=\textwidth]{figs/qt_model.pdf}
    \caption{Schematic representation of the model of the bar.}
    \label{fig:model}
\end{figure}

In \cref{fig:model} you can see the model of the bar, where:
\begin{itemize}
    \item $\lambda_V$ is the average arrival rate for VIP customers;
    \item $\lambda_N$ is the average arrival rate for ``normal'' (i.e. non-VIP) customers;
    \item $\mu_C$ is the average service rate of the cashier ($r_{cashier}$);
    \item $\mu_K$ is the average service rate of the kitchen ($r_{kitchen}$);
    \item $\pi_C$ is the ratio of compound orders over the total, i.e. the 
        probability that an order is compound.
\end{itemize}

The cashier is modeled as a service center with two queues that are managed 
in an head-of-line-priority fashion.

The kitchen is modeled as a simple M/M/1 service center. Note that if VIP 
priority is introduced also in the kitchen, it will become identical to the 
cashier, mutatis mutandis.

\subsection{Assumptions}
\begin{itemize}
    \item No renegation: customers cannot leave the queue.
    \item No jockeying: VIP customers cannot move to the normal customers' queue.
    \item Infinite queueing space: there is no upper bound in the number of 
        customers in a queue.
\end{itemize}

\subsection{Validation?}

\subsection{Stochastic model for the exponential scenario}
\begin{align}
    E[W^C_{V}] &= \frac{\lambda_V + \lambda_N}{(\mu_C-\lambda_{V})\mu_C} \\
    E[W^C_{N}] &= \frac{\lambda_V + \lambda_N}{(\mu_C-\lambda_{V}-\lambda_N)(\mu_C-\lambda_{V})} \\
    E[R^K] &= \frac{1}{\mu_K-p_C(\lambda_{V}+\lambda_{N})}
\end{align}