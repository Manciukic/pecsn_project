\section{Introduction}

\subsection{Problem description}
TODO

\subsection{Objectives}
In this analysis, the following objectives will be investigated:
\begin{itemize}
    \item relationship between the cashier service time and the overall experienced
        response time. In particular, the minimum cashier rate that guarantees a 
        given customer QoS, in terms of maximum response time.
    \item advantages, in terms of response time, of being a VIP customer.
    \item adantages and disadvantages of introducing priority head-of-line queueing
        also in the kitchen.
    \item optimal value for the ratio between VIP and normal customers in order to achieve the best overall quality of service.
    \item demonstrate that queue lengths don't depend on the ratio between the service rates.
\end{itemize}

\subsection{Performance indexes}
In order to fulfill the above-stated objectives, the following performance 
indexes will be taken into consideration:
\begin{itemize}
    \item response times for all four categories of users (normal/VIP, 
    simple/compound order), in particular their mean and 90\textsuperscript{th} 
    percentile.
    \item average queueing time for all queues.
    \item average queue length.
\end{itemize}

\subsection{Scenarios}
The following scenarios will be taken into consideration:
\begin{itemize}
    \item constant interarrival times, constant service times;
    \item exponential distribution of interarrival and service times;
    \item ``realistic'' distribution of interarrival and service times. TODO
\end{itemize}
